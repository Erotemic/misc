\documentclass[20pt]{article}

%\usepackage[margin=2cm]{geometry}

%\usepackage[hang]{caption} 
%\usepackage[T1]{fontenc}

%\usepackage{times}
%\usepackage{verbatim} 
%\usepackage{graphicx}
\usepackage{amsmath}
\usepackage{mathrsfs} % Fancy script mathscr
\usepackage{amsthm}
\usepackage{amssymb}
%\usepackage[usenames,dvipsnames,svgnames,table]{xcolor}

%\usepackage{mathpazo}

%\usepackage{xstring}
\usepackage{xcolor}
\usepackage{mathtools}
%\usepackage{algorithm}
%\usepackage{bbm}


%\usepackage[margin=2cm]{geometry}
%\usepackage{amsmath}


\begin{document}

%\colorbox{blue!30}{blue}

%\color{white}


\newcommand{\mat}[1]{\ensuremath{\mathbf{#1}}} %Should always be capital
\newcommand{\ttime}{\scalebox{.5}{\times}} %Should always be capital
\newcommand{\scale}[1]{\scalebox{.8}{#1}} %Should always be capital



\Huge


%\begin{equation*}
%    \underset{\scale{m\times{}n}}{\mat{M}}=
%    \underset{\scale{m\times{}m}}{\mat{U}}\cdot
%    \underset{\scale{m\times{}n}}{\mat{\Sigma}}\cdot
%    \underset{\scale{n\ttime{}n}}{\mat{V}}^{\mkern-4mu\mathbf{*}}
%\end{equation*}
\begin{equation*}
    \mathbf{M}=
    \mathbf{U}\cdot
    \mathbf{\Sigma}\cdot
    \mathbf{V}^{*}
\end{equation*}

%= 
%\left(
% \begin{array}{ccccc}
%   \sigma_1\\
%    & . & & \text{\huge0}\\
%    & & .\\
%    & \text{\huge0} & & \sigma_r\\
%    & & & & 0
% \end{array}
%\right)

\end{document}

